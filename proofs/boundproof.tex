\documentclass[12pt]{article}
\usepackage{amsmath}
\begin{document}

\title{Hypothesis Testing for Sampling}
\author{Anastassia Kornilova}
\date{}
\maketitle

Let $x_1 \cdots x_N$ be a population with an unknown mean $\mu$. Suppose we pick a sample without replacement of $y_1 \cdots y_n$. We want to test if the sample sum will have the same sign as the population sum, by using a hypothesis test. Let:
$$S = \sum_{i=1}^N x_i = n\mu \ \  S_n = \sum_{j=1}^n y_j$$

Let the null hypothesis be $S < 0 $.\\
By the bound defined in Serfling in {\it Probability Inequalities for Sum In Sampling Without Replacement}:
$$ P( | S_n - S | > t ) \leq 2  \exp(-2 t^2 * N/ (n (N-n-1)(b-a)^2)) $$
where $b$ and $a$ represent bounds on the population. Flipping the bounds and expanding the absolute value:
$$ P( -t < S_n - S < t ) \geq 2  \exp(-2 t^2 * N/ (n (N-n-1)(b-a)^2)) $$
Rearranging the terms, leads to a confidence level expression for $n\mu$.
$$ P(S_n - t < S < S_n + t) \geq 2  \exp(-2 t^2 * N/ (n (N-n-1)(b-a)^2)) $$
If we want this to represent an $\alpha \% $ coincidence interval, we set the right hand side equal to $\alpha$ and solve for $t$.
$$ \alpha = 2  \exp(-2 t^2 * N/ (n (N-n-1)(b-a)^2))$$
$$ t = \sqrt{\frac{n (N-n-1)(b-1)^2}{-2N} * \log(\alpha/2)}$$
Using this value of $t$, we can test if the $\alpha \%$ confidence interval contains $S$, the hypothesized value. If it doesn't, then performing the significance test will lead to a p-value $< \frac{100 - \alpha}{100}$. The value of $S$ is unknown, so we can not test if $S_n - t < S$. If $S_n - t > 0$, then $S_n - t > S$. Since $S$ can represent any negative value, it follows that if $S_n - t > 0$, we can reject $H_0$.
\end{document}
